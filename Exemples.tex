\documentclass[a4paper,12pt]{article}




\usepackage[francais]{babel}
\usepackage[utf8]{inputenc}
%\usepackage[latin1]{inputenc}
\usepackage[T1]{fontenc}
\usepackage{amssymb,amsmath,amsthm}
\usepackage{graphics, graphicx}
\usepackage{fancyhdr}
\usepackage{a4wide}
\usepackage{multicol}
\usepackage{colortbl}
\usepackage{tikz}
\usetikzlibrary{shapes}
%\usepackage{wrapfig}



%%% ARBRE 

\usepackage{caption}
\usepackage{subcaption}
\usepackage[francais]{varioref}
\usepackage[top=2cm, bottom=2.5cm, left=1.9cm, right=1.9cm]{geometry}
\usepackage{multirow}
%\usepackage{a4wide}
\usepackage{url}
%\usepackage{times}
\usepackage{cancel}


\usepackage{tikz,pgflibraryarrows,pgffor,pgflibrarysnakes}


%\usepackage{algorithm2e}
\usepackage[french,vlined]{algorithm2e}
\usepackage{sfmath}

\pagestyle{empty}


\theoremstyle{theorem}
\newtheorem{question}{Question}

\theoremstyle{definition}
\newtheorem{quizz}{Quizz}
\newtheorem*{defi}{Défi !}


\setlength{\parindent}{0pt} 

\AtBeginDocument{
\renewcommand{\labelitemi}{\textbullet}
\renewcommand{\labelitemii}{$\circ$}
\renewcommand{\labelitemiii}{$\star$}
}


\renewcommand{\familydefault}{\sfdefault}
\renewcommand{\baselinestretch}{1.2}



%%%%%%%%%%%%%%%%%%%%%%%% 
%%% Quelques paquets g\'erant la correction
%%%%%%%%%%%%%%%%%%%%%%%%
\usepackage{comment}
%%% Pour afficher la correction
%\includecomment{correction}
%%% Pour version cachant la correction
\excludecomment{correction}  
\usepackage{xcolor}
\usepackage{framed} %%% boîtes permettant de g\'erer correctement le saut de page.
\colorlet{shadecolor}{gray!10!red!5}





\begin{document}



\vspace{.7cm}

\hrule
\vspace{.3cm}



\begin{center}

{\textbf{\Large 
Exemples\\
Arithm\'etique et Cryptographie\\
Notions-cl\'es 2 et 3}}
\vspace{.1cm}

\end{center}
\vspace{.3cm}

\hrule
\vspace{.5cm}



% \thispagestyle{fancy}

$$
a^{5b}\times a^{b+1}
$$

$$
(a^2)^{3b+1}
$$

$$(1^{15}+a^{15}\times0^{3}+b^{15})\times(b^{15}-1)$$


$$\left(10^{\left(10^{99}\right)}\right)^{10}$$

$$10^{\left(\left(10^{10}\right)^{10}\right)}$$

$$10^{\left(10^{99}\right)}\times10^{10}$$

$$\left(10^{\left(10^{10}\right)}\right)^{\left(10^{90}\right)}$$


$$3^{(3^4)}$$

$$(3^3)^4$$

$$3^{(2^3).(5^3)}$$


$$5^2\times3\times3^{12}\times5\times3^3\times3^4\times5^9$$

$$\left(9^2\right)^{3}\times\left(5^2\right)^{5}\times 25$$

$$\left(\left(5^2\times3\right)^3\right)^{\left(2^2\right)}$$

$$125^4\times27^5$$

$$\left(5^3\times9\right)^2\times3^{\left(2^{\left(2^2\right)}\right)}\times5^6$$


$$27\times\left(3^2\times(3^0)^5\times3^{(3^2)}\right)^3$$


$$n^4\times\left(n^{(2^3)}\times\left(n^0\right)^3\times n^2\right)^4$$


$$a^3\times\left(a^2\times(a^0)^3\times a^{(2^3)}\right)^4$$

$$a^5\times (a^3)^n+1^{504}\times a^{15}\times a^{-2} + (b^5)^3+ 0^{12}$$


$$\displaystyle\frac{a^{-5}\times a^2}{a^3\times a^{-7}}$$

$$\displaystyle\frac{\left(a^7\right)^3}{\left(a^{-2}\right)^{-6}}$$

$$\displaystyle\left(\frac{a^{-3}}{a^5}\right)^7$$

$$\left(a^{-2}\times a^7\right)^{-3}$$






\vspace{.5cm}

\hrule
\vspace{.5cm}

\begin{itemize}

\item Justifiez que $7^{30} \equiv 1 \pmod{31}$. Déduisez-en la valeur de
$7^{902} \pmod{31}$



\item Sachant que $a$ est divisible par $13$ et que le reste de la division euclidienne de $b$ par $13$ est $1$, calculez $A=(a^{14}\times b^{12} + b^{42}+14^6)^2\times29^3 \mod\ 13$. 



\item Calculez $2^6\mod 9$, $12^2\mod9$ et $8^2\mod9$. Calculez ensuite $A\mod9$ avec
$A=2^{6m}+12^{m+2}+8^{610}$
pour tout entier $m\geq1$.



\item Pour quelles valeurs de $m$ le nombre $2^{4m+2}+4^{m}$ est-il divisible par 5? Justifier.


\item Calculez les valeurs
$5^n \mod\ 13$ pour $n$ allant de 1 à 5. Déduisez-en la valeur de
$5^{247} \mod\ 13$ en justifiant votre réponse.


\item Sachant que $a$ est divisible par 5, et que le reste de la division euclidienne de $b$ par 5 vaut 2, calculez $A=(b^{42}+a^{874 m +38}\times 8 + 2^5)\times b^4$ modulo 5.


\item Montrer que, pour tout $n\in{\mathbb N}$, le nombre $14^{2n+1}+6^{4n}+22^{n}+75$ est un multiple de 7.



\item Soit $m = 123456789$. Calculez $E = 2^{(m+3)} + 5^{(2.m)} + 7^{(2.m+1)} \mod\ 8$. Calculez d'abord $2^3\mod\ 8$, $5^2\mod\ 8$ et $7^2\mod\ 8$.



\end{itemize}




\end{document} 







\begin{quizz}

\end{quizz}
\begin{correction}
\begin{shaded}
\textbf{R\'eponse.}\ 
   

aa
\end{shaded}
\end{correction}
\vspace{.5cm}









\begin{defi}

\end{defi}
\begin{correction}
\begin{shaded}
\textbf{R\'eponse.}\ 
   

aa
\end{shaded}
\end{correction}
\vspace{.5cm}

